
\section{Project Structure}
The subsection introduces the project structure , pages directory structure and server endpoint.
\subsection{Overview Structure}
\begin{enumerate}
	\item assets/: This folder is for uncompiled assets such as public js and css files.
	
	\item components/: This folder is for Vue.js components. Components are reusable elements that you can use across different parts of your application.
	
	\item layouts/: This folder contains layout components. Layouts are used to define the structure of your pages. For instance, you might have a default layout and a dashboard layout.
	
	\item pages/: This folder contains your application's views and routes. Each file in this directory automatically becomes a route in your app.
	
	\item plugins/: This folder is for JavaScript plugins that need to be run before instantiating the root Vue.js application.
	
	\item public/: This folder contains static files used to serve publicly that you want to be publicly.
	
	\item server/: This folder is used for server-side related code, such as API routes or middleware.
	
	\item SQL/: This folder seems to be intended for database-related files, which might include SQL scripts or database schema files.
\end{enumerate}
\subsection{Page directory Structure}
\subsection{Server Endpoint}
The website only shows the information including the employees, service and projects, thus all APIs are GET methods.
\begin{enumerate}
	\item GET /api/employee?id=\{id\}: Fetches the full content of a specific exployee identified by its ID. If no ID is provided, it returns all employees that match the query condition.
	Example:
	\begin{verbatim}
	{
		"data": [
		{
			"id": 1,
			"name": "Sofia Bianchi ",
			"role": "Director",
			"pic": "/sofia.jpg",
			"cv": "Sofia Bianchi, our esteemed Director...",
			"service": [],
			"project": []
		}
		]
	}
	\end{verbatim}
	\item GET /api/project?id=\{id\}: Fetches the full content of a specific project identified by its ID. If no ID is provided, it returns all projects that match the query condition. 
	Example:
	\begin{verbatim}
	{
		data:[{	
			id:1,
			name:"EmpowerHer",
			description:"The EmpowerHer program i… resilient communities.",
			tag:"Economic Empowerment Program",
			"pic": "/EmpowerHer.jpg",
		}]
	}
	\end{verbatim}
	\item GET /api/service?id=\{id\}: Fetches the full content of a specific service identified by its ID. If no ID is provided, it returns all services that match the query condition.
	Example:
	\begin{verbatim}
	{
		"data": [
		{
			"id": 1,
			"name": "Work",
			"description": "Our Work Services...",
			"availability": "from 9:00 am to 5:00 pm",
			"pic": "...",
			"tag": "..."
		}
		]
	}
	\end{verbatim}
		\item GET /api/testimonial?serviceID=\{id\}: Fetches the full testimonial of a service identified by its ID.
	Example:
	\begin{verbatim}
		{
			"data": [
			{
				"id": 1,
				"serviceID": 1,
				"comment": "Anemone empowered me to restart my career safely",
				"name": "Arianna",
				"age": 24
			},
			]
		}
	\end{verbatim}
\end{enumerate}