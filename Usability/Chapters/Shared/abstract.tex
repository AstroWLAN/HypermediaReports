% Document abstract 
\pagenumbering{arabic}
\section{Abstract}
The aim of this document is to carry out a usability evaluation of the website
\begin{center}
    \url{https://www.unicef.org}
\end{center}
We can distinguish two main areas: inspection and user testing. 
\subsection{Inspection}
For the site inspection, we selected twelve heuristics from the MILE set in addition to those from Nielsen.
Each team member performed individual analyses and the results were discussed together to identify the most critical issues.
We chose not to average the results of the individual scores, recognising that details missed in the individual reviews could significantly alter the previously provided scores.
The joint assessment did not reveal any significant critical issues at the site; however, there is still considerable room for improvement in certain areas.
In particular, our evaluation revealed a prevailing sense of disorientation in navigation, coupled with a lack of consistency.

\subsection{Testing}
To streamline the tests, we created a series of high-level tasks without going into detailed instructions on how to perform them. 
We meticulously recorded the time taken by each participant to complete each task, and then set thresholds to classify test results as either successful or unsuccessful. 
For logistical reasons, we conducted some tests remotely rather than in person.
Participants were unable to interact with us in any way. Our focus was solely on gathering personal and non-personal feedback on the usability of the site.
The majority of the tests were conducted using a PC, with some testers specifically tasked to evaluate the site from a mobile device.
The tests confirmed many of our personal impressions and shed light on some aspects we had not previously considered, such as the poor experience of using the site from a mobile device.