% User testing results 
\section{Conclusions}

\subsection{Results Comparison}
The team evaluation, which used established heuristics to evaluate overall usability, highlighted several areas for improvement, such as greater attention to detail and minimizing unnecessary repetition.
Despite the assignment of shared scores, there was a widespread sense of dissatisfaction among the evaluators due to the numerous small imperfections found.
On the other hand, the testing phase provided specific insights into users' experiences with activities such as donating, subscribing, and browsing.
Users have experienced difficulties and frustrations, particularly with certain activities and with the mobile version of the website.
A desire for a minimalist design approach and better navigation design was also expressed.
Comparing the results obtained using the two assessment methods for the UNICEF website highlights several key aspects and differences in the results:\\

\textbf{\color{unicefGreen}{Similiarties}}
\begin{enumerate}
\item \textbf{Observations of Non-Trivial Tasks} \\
Both evaluations highlighted that certain tasks were perceived as non-trivial and challenging by users. 
This indicates a consistent finding regarding the difficulty of completing specific actions on the website.

\item \textbf{Criticism of Mobile Version} \\
Subjects from both evaluations expressed dissatisfaction with the mobile version of the website. 
They found it less user-friendly compared to the desktop version, suggesting a common issue with mobile usability.

\item \textbf{Desire for Minimalist Design} \\
Subjects from both evaluations expressed a desire for a minimalist design approach, focusing on visual storytelling and reducing information overload. 
This indicates a shared preference for a streamlined and visually engaging website layout.
\end{enumerate}

\textbf{\color{unicefRed}{Differences}}
\begin{enumerate}
\item  \textbf{Scope of Evaluation} \\
While the group evaluation focused on overall usability using established heuristics, the testing phase targeted specific tasks performed by users. 
This led to differences in the level of detail and scope of feedback provided.

\item \textbf{Methodological Approach} \\
The group evaluation utilized a collaborative discussion to arrive at shared scores, while the user testing gathered feedback directly from users. 
This difference in methodology influenced the nature of the feedback obtained and the depth of insights provided.

\item \textbf{Specific Recommendations} \\
While both evaluations offered recommendations for improvement, the user testing provided more specific insights into user experiences with tasks such as donation, subscription, and navigation. 
This granularity allowed for targeted suggestions for enhancing user interactions on the website.
\end{enumerate}

\newpage
\subsection{Redesign Suggestions}
Despite its overall efficiency, the UNICEF website faces usability challenges, including difficulties with complex tasks, dissatisfaction with the mobile version's user-friendliness compared to the desktop and a preference for a minimalist design that prioritizes visual storytelling and reduces information overload.
To enhance the website, we recommend the following improvements:

\begin{enumerate}
	\item \textbf{Enhance Navigation}\\
	Ensure intuitive navigation by maintaining clear and visible breadcrumbs, helping users understand their location on the site and how to return to previous or higher-level pages easily.
	
	\item \textbf{Improve Multilingual Support}\\
	Enhance translations quality and localization to make content feel natural and engaging across all supported languages.
	
	\item \textbf{Refine Search and Assistance}\\
	Upgrade the search function for more relevant results, particularly for essential tasks like locating contact details or specific initiatives. 
	Incorporate auto-suggestions and filters for a streamlined search experience.

	\item \textbf{Optimize for Mobile}\\
	Make the mobile site more user-friendly, aligning its usability closer to the desktop version.

	\item \textbf{Incorporate User Feedback}\\
	Establish a mechanism for collecting user feedback, allowing for continuous identification and addressal of usability concerns.
\end{enumerate}

\subsection{Personal Reflefctions}
Heuristic evaluation provides a quick and cost-effective way to identify potential usability problems, user testing offers valuable insights into actual user behavior and reveals unexpected problems that expert evaluations may miss.
Each method offers distinct benefits and, when used together, create a more holistic approach to identifying and addressing usability issues.
