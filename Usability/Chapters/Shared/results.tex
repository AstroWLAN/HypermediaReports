% User testing results 
\section{Results}

\subsection{Comparison of the results}
The group evaluation, which utilized established heuristics to assess overall
usability, highlighted several areas for improvement, including attention to detail,
minimizing unnecessary repetition, and incorporating widely used languages like
Arabic. Despite assigning shared scores, there was a pervasive sense of
dissatisfaction among the evaluators due to numerous imperfections identified.

On the other hand, the individual evaluation provided specific insights into user
experiences with tasks such as donation, subscription, and navigation. Users
encountered difficulties and frustrations, particularly with certain tasks and the
mobile version of the website. The desire for a minimalist design approach and
better navigation design was also expressed

The comparison of the results achieved using the two evaluation methods for the UNICEF website—group evaluation and individual evaluation—highlights several key aspects and differences in the findings:

Similarities in Findings
\begin{enumerate}
\item Observations of Non-Trivial Tasks: Both evaluations noted that certain
tasks were perceived as non-trivial and challenging by users. This indicates a
consistent finding regarding the difficulty of completing specific actions on
the website.

\item Criticism of Mobile Version: Users from both evaluations expressed
dissatisfaction with the mobile version of the website. They found it less
user-friendly compared to the desktop version, suggesting a common issue
with mobile usability.

\item Desire for Minimalist Design: Users from both evaluations expressed a
desire for a minimalist design approach, focusing on visual storytelling and
reducing information overload. This indicates a shared preference for a
streamlined and visually engaging website layout.
\end{enumerate}

Differences in Findings
\begin{enumerate}
\item  Scope of Evaluation: While the group evaluation focused on overall usability using established heuristics, the individual evaluation targeted specific tasks performed by users. This led to differences in the level of detail and scope of feedback provided.

\item Methodological Approach: The group evaluation utilized a collaborative discussion to arrive at shared scores, while the individual evaluation gathered feedback directly from users. This difference in methodology influenced the nature of the feedback obtained and the depth of insights Provided.

\item Specific Recommendations: While both evaluations offered recommendations for improvement, the individual evaluation provided more specific insights into user experiences with tasks such as donation, subscription, and navigation. This granularity allowed for targeted suggestions for enhancing user interactions on the website.
\end{enumerate}


\subsection{Suggestions for redesign}
Despite its overall efficiency, the UNICEF website faces notable usability challenges, including difficulties with complex tasks, dissatisfaction with the mobile version's user-friendliness compared to the desktop, and a preference for a minimalist design that prioritizes visual storytelling and reduces information overload.
To enhance the website, we recommend the following improvements:

\begin{enumerate}
	\item Enhance Navigation: Ensure intuitive navigation by maintaining clear and visible breadcrumbs, helping users understand their location on the site and how to return to previous or higher-level pages easily.
	
	\item Improve Multilingual Support: Enhance translation quality and localization to make content feel natural and engaging across all supported languages.
	
	\item Refine Search and Assistance: Upgrade the search function for more relevant results, particularly for essential tasks like locating contact details or specific initiatives. Incorporate auto-suggestions and filters for a streamlined search experience.
	\item Optimize for Mobile: Make the mobile site more user-friendly, aligning its usability closer to the desktop version.
	\item Incorporate User Feedback: Establish a mechanism for collecting user feedback, allowing for continuous identification and addressal of usability concerns.
\end{enumerate}

\subsection{What we learnt from personal observations}
Heuristic evaluation provides a quick, cost-effective way to identify potential usability issues, user testing offers invaluable insights into actual user behavior and reveal unexpected issues that expert evaluations might miss. Each method offers distinct advantages and, when used together, they create a more holistic approach to identifying and addressing usability issues.
