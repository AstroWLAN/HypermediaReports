
\section{Evaluator : Yan} 
\subsection{Nielsen’s heuristics scores}
\begin{table}[htp!]
	\centering
	\begin{tabular}{ |l|l|c| }
		\hline
		\textbf{Code} & \textbf{Description} & \textbf{Score}\\
		\hline
		\textbf{H1} & Visibility of system status & \textbf{\color{unicefOrange}{3}}\\
		\hline
		\textbf{H2} & Match between system and the real world & \textbf{\color{unicefGreen}{4}}\\
		\hline
		\textbf{H3} & User control and freedom & \textbf{\color{unicefGreen}{4}}\\
		\hline
		\textbf{H4} & Consistency and standards & \textbf{\color{unicefGreen}{4}}\\
		\hline
		\textbf{H5} & Error prevention & \textbf{\color{unicefGreen}{4}}\\
		\hline
		\textbf{H6} & Recognition rather than recall & \textbf{\color{unicefGreen}{4}}\\
		\hline
		\textbf{H7} & Flexibility and efficiency of use & \textbf{\color{unicefOrange}{3}}\\
		\hline
		\textbf{H8} & Aesthetic and minimalist design & \textbf{\color{unicefGreen}{4}}\\
		\hline
		\textbf{H9} & Help users recognize, diagnose and recover from errors & \textbf{\color{unicefGreen}{4}}\\
		\hline
		\textbf{H10} & Help and documentation & \textbf{\color{unicefGreen}{5}}\\
		\hline
	\end{tabular}
	\caption{\textbf{Nielsen}'s heuristics scores}
\end{table}
% MILE Heuristics Evaluation
\subsection{Mile’s heuristics scores}
\begin{table}[htp!]
	\centering
	\begin{tabular}{ |c|l|l|c| }
		\hline
		\textbf{Group} & \textbf{Code} & \textbf{Description} & \textbf{Score}\\
		\hline
		\multirow{5}{*}{\textbf{Navigation}} & \textbf{M1} & Interaction consistency & \textbf{\color{unicefGreen}{4}}\\
		\cline{2-4}
		& \textbf{M2} & Group navigation & \textbf{\color{unicefGreen}{4}}\\
		\cline{2-4}
		& \textbf{M3} & Structural Navigation & \textbf{\color{unicefGreen}{4}}\\
		\cline{2-4}
		& \textbf{M4} & Semantic Navigation & \textbf{\color{unicefGreen}{5}}\\
		\cline{2-4}
		& \textbf{M5} & Landmarks & \textbf{\color{unicefGreen}{4}}\\
		\hline
		\textbf{Content} & \textbf{M6} & Information overload & \textbf{\color{unicefGreen}{4}}\\
		\hline
		\multirow{6}{*}{\textbf{Presentation}} & \textbf{M7} & Text layout & \textbf{\color{unicefGreen}{5}}\\
		\cline{2-4}
		& \textbf{M8} & Interaction placeholders-semiotics & \textbf{\color{unicefGreen}{4}}\\
		\cline{2-4}
		& \textbf{M9} & Interaction placeholders-consistency & \textbf{\color{unicefGreen}{4}}\\
		\cline{2-4}
		& \textbf{M10} & Spatial allocation & \textbf{\color{unicefGreen}{4}}\\
		\cline{2-4}
		& \textbf{M11} & Consistency of Page Structure & \textbf{\color{unicefGreen}{4}}\\
		\cline{2-4}
		& \textbf{M12} & Coherence in page layout & \textbf{\color{unicefGreen}{5}}\\
		\hline
	\end{tabular}
	\caption{\textbf{MILE} heuristics scores}
\end{table}

\subsection{Comments on Nielsen’s heuristics:}
\subsubsection*{H1. Visibility of system status}
The website uses breadcrumbs on some pages, but not on others. E.g., https://www.unicef.org/blog doesn't apply the breadcrumbs, which possibly makes users confused.

\subsubsection*{H2. Match between system and the real world}
The website provides multiple languages, including Chinese, English, French, and others, respecting cultural conventions and enhancing accessibility.

\subsubsection*{H3. User control and freedom }
The website grants users considerable flexibility and control when browsing and engaging with its content.

\subsubsection*{H4. Consistency and standards}
Most web pages follow standards, For example: icons, links, color codes, font sizes and font family, picture. 
Only a mistake found, the top banner in the home page(https://www.unicef.org/) show the languages support, but a link button navigating to the home in the page( https://www.unicef.org/innocenti/projects/changing-childhood)
\begin{figure}[h]
	\centering
	\includegraphics[width=0.5\textwidth]{Images/yan_h4_1.png}
	\caption{https://www.unicef.org/}
	\label{fig:h4_1}
\end{figure}
\begin{figure}[h]
	\centering
	\includegraphics[width=0.5\textwidth]{Images/yan_h4_2.png}
	\caption{https://www.unicef.org/innocenti/projects/changing-childhood}
	\label{fig:h4_2}
\end{figure}

\subsubsection*{H5. Error prevention }
The website has strong error prevention like inspections of valid email and phone format in donation page, and email format in sign-in page. 

\subsubsection*{H6. Recognition rather than recall}
The website did a great job on the heuristic. It applies functional icons like search bar to represent actions, rather than make user to remember type commands. Plus, it offers options in the donation page and language menu, reducing the difficulty of use. 
\begin{figure}[H]
	\centering
	\includegraphics[width=0.5\textwidth]{Images/yan_h6.png}
	\caption{Options in the donation page}
	\label{fig:h6}
\end{figure}

\subsubsection*{H7. Flexibility and efficiency of use}
The website doesn't provide many web accelerators.

\subsubsection*{H8. 	Aesthetic and minimalist design}
The website has a user-friendly interface that maintains simplicity and visual elegance.

\subsubsection*{H9. Help users recognize, diagnose and recover from errors}
The website displays a 404 page if users input the wrong URL. Also, on the donation and sign-in pages, it verifies the valid format of an email
 
\subsubsection*{H10. Help and documentation }
The website offers all kinds of documents on  https://open.unicef.org/documents. 
	 
\subsection{Comments on Mile’s heuristics:}
\subsubsection*{M1. Group Navigation}
\subsubsection*{M2. Structural Navigation}
\subsubsection*{M3. Interaction Consistency}
\subsubsection*{M4. Semantic Navigation}
\subsubsection*{M5. Landmarks}
\subsubsection*{M6. Information overload}
\subsubsection*{M7. Text Layout}
\subsubsection*{M8. Interaction Placeholders – Semiotics}
\subsubsection*{M9. Interaction Placeholders – Consistency}
\subsubsection*{M10. Spatial Allocation}
\subsubsection*{M11. Consistency of Page Structure}
\subsubsection*{M12. Coherence in page layout}
 	 
 	 



