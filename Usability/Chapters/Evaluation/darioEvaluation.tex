% Dario's website evaluation 
\section{Dario's Evaluation}
\subsection{Scores}
Outlined below are the evaluations provided by Dario Crippa for the Nielsen and MILE heuristics carefully selected by the group to analyze the website.\\
% Nielsen's Heuristics Evaluation
\begin{table}[htp!]
    \centering
    \begin{tabular}{ |l|l|c| }
        \hline
        \textbf{Code} & \textbf{Description} & \textbf{Score}\\
        \hline
        \textbf{H1} & Visibility of system status & \textbf{\color{unicefOrange}{3}}\\
        \hline
        \textbf{H2} & Match between system and the real world & \textbf{\color{unicefGreen}{5}}\\
        \hline
        \textbf{H3} & User control and freedom & \textbf{\color{unicefRed}{2}}\\
        \hline
        \textbf{H4} & Consistency and standards & \textbf{\color{unicefGreen}{5}}\\
        \hline
        \textbf{H5} & Error prevention & \textbf{\color{unicefGreen}{4}}\\
        \hline
        \textbf{H6} & Recognition rather than recall & \textbf{\color{unicefGreen}{5}}\\
        \hline
        \textbf{H7} & Flexibility and efficiency of use & \textbf{\color{unicefOrange}{3}}\\
        \hline
        \textbf{H8} & Aesthetic and minimalist design & \textbf{\color{unicefOrange}{3}}\\
        \hline
        \textbf{H9} & Help users recognize, diagnose and recover from errors & \textbf{\color{unicefGray}{n.a}}\\
        \hline
        \textbf{H10} & Help and documentation & \textbf{\color{unicefGreen}{4}}\\
        \hline
    \end{tabular}
    \caption{\textbf{Nielsen}'s heuristics scores}
\end{table}
% MILE Heuristics Evaluation
\begin{table}[htp!]
    \centering
    \begin{tabular}{ |c|l|l|c| }
        \hline
        \textbf{Group} & \textbf{Code} & \textbf{Description} & \textbf{Score}\\
        \hline
        \multirow{5}{*}{\textbf{Navigation}} & \textbf{M1} & Interaction consistency & \textbf{\color{unicefRed}{2}}\\
        \cline{2-4}
        & \textbf{M2} & Group navigation & \textbf{\color{unicefGreen}{4}}\\
        \cline{2-4}
        & \textbf{M3} & Navigation support & \textbf{\color{unicefGreen}{4}}\\
        \cline{2-4}
        & \textbf{M4} & User control & \textbf{\color{unicefOrange}{3}}\\
        \cline{2-4}
        & \textbf{M5} & Error prevention & \textbf{\color{unicefGreen}{4}}\\
        \hline
        \textbf{Content} & \textbf{M6} & Information overload & \textbf{\color{unicefOrange}{3}}\\
        \hline
        \multirow{6}{*}{\textbf{Presentation}} & \textbf{M7} & Text layout & \textbf{\color{unicefGreen}{5}}\\
        \cline{2-4}
        & \textbf{M8} & Interaction placeholder semiotics & \textbf{\color{unicefGreen}{4}}\\
        \cline{2-4}
        & \textbf{M9} & Interaction placeholder consistency & \textbf{\color{unicefGreen}{4}}\\
        \cline{2-4}
        & \textbf{M10} & Spatial allocation & \textbf{\color{unicefOrange}{3}}\\
        \cline{2-4}
        & \textbf{M11} & Consistency of the page structure & \textbf{\color{unicefGreen}{4}}\\
        \cline{2-4}
        & \textbf{M12} & Coherence in page layout & \textbf{\color{unicefGreen}{5}}\\
        \hline
    \end{tabular}
    \caption{\textbf{MILE} heuristics scores}
\end{table}
% Comments 
\newpage
\subsection{Comments}
This section offers clear and concise explanations and justifications for the scores assigned to the previously presented heuristics.
% Nielsen's Heuristics Comments
\subsubsection{Nielsen's Heuristics}
\begin{description}
    \item {\textbf{H1} \color{unicefGray}{Visibility of the system status}}\\
    While breadcrumbs provide clear and intuitive navigation in some sections, other areas feel a bit chaotic. 
    There are moments where users might feel a bit lost.
    \item {\textbf{H2} \color{unicefGray}{Match between system and the real world}}\\
    The website effectively communicates in the users' native language and the organization of information is both logical and clear.
    \item {\textbf{H3} \color{unicefGray}{User control and freedom}}\\
    In the donation section an user is not able to modify the donation amount without losing the inserted personal information.
    \item {\textbf{H4} \color{unicefGray}{Consistency and standards}}\\
    All aspects are transparent and easily comprehensible devoid of any ambiguities.
    \item {\textbf{H5} \color{unicefGray}{Error prevention}}\\
    The website effectively guides users through the process of entering personal information in the donation section. 
    However, it remains possible for users to input meaningless information.
    Additionally a confirmation popup is triggered when a custom donation amount is entered.
    \item {\textbf{H6} \color{unicefGray}{Recognition rather than recall}}\\
    Using the website proficiently requires no mnemonic work. It's designed for intuitive navigation and ease of use.
    \item {\textbf{H7} \color{unicefGray}{Flexibility and efficiency of use}}\\
    While the website offers shortcuts like the share button for quick sharing on major social media platforms, the navigation remains relatively static and there are no options for customization.
    \item {\textbf{H8} \color{unicefGray}{Aesthetic and minimalist design}}\\
    The website adopts a minimalistic approach, yet its aesthetic appeal may not be particularly pronounced. 
    Repetitions, such as the presence of multiple donation buttons on the page, are noticeable.
    \item {\textbf{H9} \color{unicefGray}{Help users recognize, diagnose and recover from errors}}\\
    This aspect cannot be tested deeply cause the website is essentially static and lacks of interactive features.
    \item {\textbf{H10} \color{unicefGray}{Help and documentation}}\\
    In the footer, users can access various links to gather useful information. 
    Both FAQs and documentation are available in multiple languages for enhanced accessibility.
\end{description}
% MILE Heuristics Comments
\newpage
\subsubsection{MILE Heuristics}
\begin{description}
    \item {\textbf{M1} \color{unicefGray}{Interaction consistency}}\\
    Pages with similar content demonstrate varying navigation approaches
    \item {\textbf{M2} \color{unicefGray}{Group navigation}}\\
    Navigation throughout the website is seamless regardless of the user's position but the menus appear slightly overcrowded.
    \item {\textbf{M3} \color{unicefGray}{Navigation support}}\\
    Discovering the various components of a topic is straightforward.
    \item {\textbf{M4} \color{unicefGray}{User control}}\\
    While certain areas of the website excel in implementing interactive features others may feel somewhat unintuitive.
    \item {\textbf{M5} \color{unicefGray}{Error prevention}}\\
    Every section of the webpage is easily accessible through a simple and effective interface.
    \item {\textbf{M6} \color{unicefGray}{Information overload}}\\
    In certain sections, like the menus, the website may induce a sense of information overload.
    \item {\textbf{M7} \color{unicefGray}{Text layout}}\\
    Nothing is left to chance in the text layout, with a clear and concise structure.
    \item {\textbf{M8} \color{unicefGray}{Interaction placeholder semiotics}}\\
    The website features a modest number of interactive elements, each conveying its functional purpose clearly and operating smoothly.
    \item {\textbf{M9} \color{unicefGray}{Interaction placeholder consistency}}\\
    The website maintains a consistent approach to interaction placeholders.
    \item {\textbf{M10} \color{unicefGray}{Spatial allocation}}\\
    I find the distribution of related elements on the page adequate, but personally, I believe that certain parts of the website require redesigning to enhance the user experience. 
    Overall, the website gives me a sense of overcrowding, suggesting that spatial organization could be improved to alleviate this issue.
    \item {\textbf{M11} \color{unicefGray}{Consistency of the page structure}}\\
    All pages adhere to a similar structure, with minor justified variations.
    \item {\textbf{M12} \color{unicefGray}{Coherence in page layout}}\\
    The website's layout is coherent and consistent, with a clear and logical structure.
\end{description}


