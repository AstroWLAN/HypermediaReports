\documentclass{article}
\usepackage{multirow}
\usepackage{xcolor}
\definecolor{unicefOrange}{RGB}{255, 128, 0}
\definecolor{unicefGreen}{RGB}{0, 204, 0}
\definecolor{unicefRed}{RGB}{255, 0, 0}
\definecolor{unicefGray}{RGB}{128, 128, 128}

\begin{document}

% Sanaz's website evaluation
\section{Sanaz's Evaluation}
\subsection{Scores}
Outlined below are the evaluations provided by Sanaz for the Nielsen and MiLe heuristics carefully selected by the group to analyze the website.\\
% Nielsen's Heuristics Evaluation
\begin{table}[htp!]
    \centering
    \begin{tabular}{ |l|l|c| }
        \hline
        \textbf{Code} & \textbf{Description} & \textbf{Score}\\
        \hline
        \textbf{H1} & Visibility of system status & \textbf{\color{unicefOrange}{3}}\\
        \hline
        \textbf{H2} & Match between system and the real world & \textbf{\color{unicefGreen}{4}}\\
        \hline
        \textbf{H3} & User control and freedom & \textbf{\color{unicefOrange}{3}}\\
        \hline
        \textbf{H4} & Consistency and standards & \textbf{\color{unicefGreen}{4}}\\
        \hline
        \textbf{H5} & Error prevention & \textbf{\color{unicefGreen}{4.5}}\\
        \hline
        \textbf{H6} & Recognition rather than recall & \textbf{\color{unicefGreen}{4.5}}\\
        \hline
        \textbf{H7} & Flexibility and efficiency of use & \textbf{\color{unicefOrange}{3.5}}\\
        \hline
        \textbf{H8} & Aesthetic and minimalist design & \textbf{\color{unicefGreen}{4}}\\
        \hline
        \textbf{H9} & Help users recognize, diagnose and recover from errors & \textbf{\color{unicefGreen}{4.5}}\\
        \hline
        \textbf{H10} & Help and documentation & \textbf{\color{unicefGreen}{4}}\\
        \hline
    \end{tabular}
    \caption{\textbf{Nielsen}'s heuristics scores}
\end{table}
% MiLe Heuristics Evaluation
\begin{table}[htp!]
    \centering
    \begin{tabular}{ |c|l|l|c| }
        \hline
        \textbf{Group} & \textbf{Code} & \textbf{Description} & \textbf{Score}\\
        \hline
        \multirow{5}{*}{\textbf{Navigation}} & \textbf{M1} & Interaction consistency & \textbf{\color{unicefGreen}{4}}\\
        \cline{2-4}
        & \textbf{M2} & Group navigation & \textbf{\color{unicefGreen}{4.5}}\\
        \cline{2-4}
        & \textbf{M3} & Navigation support & \textbf{\color{unicefGreen}{5}}\\
        \cline{2-4}
        & \textbf{M4} & User control & \textbf{\color{unicefOrange}{3}}\\
        \cline{2-4}
        & \textbf{M5} & Error prevention & \textbf{\color{unicefGreen}{4}}\\
        \hline
        \textbf{Content} & \textbf{M6} & Information overload & \textbf{\color{unicefGreen}{4}}\\
        \hline
        \multirow{6}{*}{\textbf{Presentation}} & \textbf{M7} & Text layout & \textbf{\color{unicefGreen}{4}}\\
        \cline{2-4}
        & \textbf{M8} & Interaction placeholder semiotics & \textbf{\color{unicefGreen}{5}}\\
        \cline{2-4}
        & \textbf{M9} & Interaction placeholder consistency & \textbf{\color{unicefGreen}{4.5}}\\
        \cline{2-4}
        & \textbf{M10} & Spatial allocation & \textbf{\color{unicefGreen}{4.5}}\\
        \cline{2-4}
        & \textbf{M11} & Consistency of the page structure & \textbf{\color{unicefGreen}{4}}\\
        \cline{2-4}
        & \textbf{M12} & Coherence in page layout & \textbf{\color{unicefGreen}{5}}\\
        \hline
    \end{tabular}
    \caption{\textbf{MiLe} heuristics scores}
\end{table}
% Comments 
\newpage
\subsection{Comments}
This section offers clear and concise explanations and justifications for the scores assigned to the previously presented heuristics.
% Nielsen's Heuristics Comments
\subsubsection{Nielsen's Heuristics}
\begin{description}
    \item {\textbf{H1} \color{unicefGray}{Visibility of system status}}\\
    The website employs path-based breadcrumbs, allowing users to understand their current location within the site's hierarchy. However, if a user navigates through a high number of steps, there might be usability and visualization problems.
    \item {\textbf{H2} \color{unicefGray}{Match between system and the real world}}\\
    The website effectively uses language familiar to users and follows real-world conventions, enhancing user understanding and usability.
    \item {\textbf{H3} \color{unicefGray}{User control and freedom}}\\
    The UNICEF website offers users ample control and freedom in navigating and interacting with its content. Users can easily backtrack or correct actions without constraints, allowing for flexible exploration of different sections. By minimizing interruptions and providing autonomy. (Backtrack in the donation part has a problem.)
    \item {\textbf{H4} \color{unicefGray}{Consistency and standards}}\\
    The website generally maintains consistency in language and navigation. However, there might be instances where users wonder if different terms or actions mean the same thing, reducing consistency slightly.\\
    Examples:
    \begin{itemize}
        \item \textbf{Consistent Terminology:}
        \begin{itemize}
            \item The donation page consistently uses terms like "donate," "contribute," or "support" to refer to the action of giving financial assistance.
        \end{itemize}
        \item \textbf{Consistent Actions:}
        \begin{itemize}
            \item Clicking on a donation amount immediately takes users to the payment process, maintaining a consistent action-reaction pattern.
            \item Buttons or links for submitting donation information consistently lead users to a confirmation page or thank-you message, confirming their contribution.
        \end{itemize}
        \item \textbf{Platform Conventions:}
        \begin{itemize}
            \item Navigation menus and buttons behave in a manner consistent with typical web browsing experiences. For instance, the "Donate" button is prominently displayed and behaves similarly to other clickable elements on the website.
            \item Forms for entering payment information follow standard design conventions, such as using recognizable input fields and validation messages.
        \end{itemize}
    \end{itemize}
    \item {\textbf{H5} \color{unicefGray}{Error prevention}}\\
    \textbf{Homepage Navigation:} The website effectively prevents form submission with incomplete data, ensuring accurate information.\\
    \textbf{Donation Process:} Disabling the “Donate” button after the first click prevents accidental double donations.\\
    \textbf{Password Reset:} Setting an expiration time for password reset links enhances security.\\
    \textbf{File Uploads:} Validating file types and sizes prevents issues.\\
    \textbf{Confirmation Dialogs:} Displaying confirmation dialogs before irreversible actions is essential.\\
    Overall, the UNICEF website demonstrates strong error prevention practices.
    \item {\textbf{H6} \color{unicefGray}{Recognition rather than recall}}\\
    \textbf{Search Engine Results:} The website displays relevant search results, reducing the need for users to recall specific URLs.\\
    \textbf{Navigation Menus:} Clear menu options guide users without requiring them to remember page names.\\
    \textbf{Donation Amount Selection:} Predefined donation amounts simplify decision-making.\\
    \textbf{Language Selection:} A visible language dropdown helps users recognize their preferred language.\\
    \textbf{Contact Information:} Prominent contact details facilitate communication.
\end{description}
% Nielsen's Heuristics Comments (continued)
\begin{description}
    \item {\textbf{H7} \color{unicefGray}{Flexibility and efficiency of use}}\\
    UNICEF’s website effectively balances flexibility for different user levels and provides visible cues for efficient interactions. Further customization options could enhance efficiency even more.\\
    \textbf{For example,} When users search for specific content (e.g., “child vaccination programs”) on the UNICEF website, the search results page displays a list of relevant articles, reports, and resources.\\
    UNICEF’s navigation menus prominently feature categories such as “Programs,” “Get Involved,” and “Donate.” These options are visible across the site.
\end{description}
% Nielsen's Heuristics Comments (continued)
\begin{description}
    \item {\textbf{H8} \color{unicefGray}{Aesthetic and minimalist design}}\\
    The homepage banner effectively captures the user's attention with impactful imagery and concise messaging, without overwhelming them with excessive text or graphics. This minimalist approach ensures that key information is communicated clearly and effectively, enhancing user engagement and reinforcing UNICEF's mission and brand identity.\\
    Additionally, the placement of navigation elements and call-to-action buttons is strategic, ensuring that users can easily access important sections of the website without unnecessary clutter.
\end{description}
% Nielsen's Heuristics Comments (continued)
\begin{description}
    \item {\textbf{H9} \color{unicefGray}{Help users recognize, diagnose, and recover from errors}}\\
    Error messages are expressed clearly and suggest constructive solutions, aiding users in error recovery. For example:\\
    \textbf{Fraudulent Posts and Alerts:}
    \begin{itemize}
        \item On the UNICEF Uganda page, there’s an alert warning user about fraudulent job postings.
        \begin{itemize}
            \item Error Message: “The alert contained an outdated UNICEF logo, wrong email address, and had errors in the unpacked UNICEF acronym. UNICEF in full is United Nations Children’s Fund and NOT United Nations International Children’s Emergency Fund. All UNICEF job applications are done online and not shared via email as indicated in the fake advert.”
        \end{itemize}
    \end{itemize}
    This error message clearly identifies the issue (fraudulent job posting) and provides constructive solutions (correcting the UNICEF acronym, emphasizing online applications). Users can recognize the problem and take appropriate action.
\end{description}
% Nielsen's Heuristics Comments (continued)
\begin{description}
    \item {\textbf{H10} \color{unicefGray}{Help and documentation}}\\
    UNICEF provides valuable documentation and resources, but there’s room for further improvement in user-friendly guidance. For example:\\
    The Knowledge @ UNICEF section offers a wealth of resources and knowledge products on various topics such as water and sanitation, child protection, and nutrition. Users can access case studies, field notes, and training materials related to UNICEF’s work.
\end{description}
% MiLe Heuristics Comments
\subsubsection{MiLe Heuristics}
\begin{description}
    \item {\textbf{M1} \color{unicefGray}{Consistency of interaction}}\\
    The UNICEF website demonstrates consistent interaction patterns across its pages. Users encounter familiar elements such as menus, buttons, and links consistently throughout the site. For example: The navigation menu remains fixed at the top of the page, allowing users to access key sections regardless of their scroll position.
    \item {\textbf{M2} \color{unicefGray}{Group navigation}}\\
    The website groups related content logically, making it easier for users to find relevant information. Sections are organized into clear categories (e.g., “Child protection,” “Education,” “UNICEF in emergencies”). For example: The “Programme” section groups various initiatives, such as child survival, education, and gender, under distinct headings.
    \item {\textbf{M3} \color{unicefGray}{Structural navigation}}\\
    The site’s structure follows a hierarchical model, with main sections leading to subpages. Users can navigate deeper into specific topics. For example, The “Explore what we do” section provides a structured overview of UNICEF’s work, allowing users to explore child protection, education, and other areas.
    \item {\textbf{M4} \color{unicefGray}{Semantic navigation}}\\
    UNICEF uses clear and descriptive labels for navigation elements. Users can infer the content of each link without ambiguity. For example, the “Get involved,” “Take action,” and “UNICEF Results” sections provide semantically meaningful labels. But in the mobile platform, it's not mobile-friendly.
    \item {\textbf{M5} \color{unicefGray}{Presence of landmarks}}\\
    The website includes landmarks (such as headers, footers, and sidebars) that aid navigation. These landmarks help users orient themselves within the content. For example, the footer contains links to key resources, including UNICEF’s accessibility toolkit and information about the Sustainable Development Goals.
\end{description}
% MiLe Heuristics Comments (continued)
\begin{description}
    \item {\textbf{M6} \color{unicefGray}{Content (Information overload)}}\\
    The UNICEF website strikes a commendable balance between comprehensive information and avoiding overwhelming users. For example, the homepage provides concise summaries of key initiatives (e.g., child protection, education, health) with links to explore further. Detailed information is available on dedicated pages.
    \item {\textbf{M7} \color{unicefGray}{Text layout}}\\
    The typography on the UNICEF website is clear and legible. Headings, subheadings, and body text are appropriately styled.
    \item {\textbf{M8} \color{unicefGray}{Interaction placeholders (Semiotics)}}\\
    The UNICEF website effectively uses semiotics (symbols, icons, and visual cues) to guide user interactions. These symbols convey meaning without relying solely on text. For example, icons for actions like “Donate,” “Share,” and “Download” are universally recognizable and enhance usability.
    \item {\textbf{M9} \color{unicefGray}{Interaction placeholders (Consistency)}}\\
    Consistency in interaction placeholders ensures that similar actions are represented uniformly across the site. Users can predict the outcome of clicking buttons or links. Like the “Donate” button appears consistently in the header, footer, and relevant content sections.
    \item {\textbf{M10} \color{unicefGray}{Spatial allocation of content}}\\
    Content is well-organized, with a clear hierarchy. Important information is prominently placed, while secondary content is appropriately nested. For example, the homepage features impactful visuals (e.g., images of children) alongside succinct text.
    \item {\textbf{M11} \color{unicefGray}{Consistency of page structure}}\\
    UNICEF maintains a consistent layout across pages. Users can predict where to find specific elements (e.g., navigation, footer). Like the header contains the logo, menu, and search bar consistently across pages.
    \item {\textbf{M12} \color{unicefGray}{Coherence in page layouts}}\\
    Page layouts are coherent, with a logical flow. Users can follow the narrative without confusion. For example, the “Stories” section presents articles with a consistent format (title, image, summary).
\end{description}
\end{document}
